\documentclass[a4paper]{jsbook}	% 基準となるフォントサイズは11pt
\pagestyle{empty}
\usepackage{booktabs}
\usepackage{tabularx}	% tabularx(横幅を指定した表)を使う
\usepackage{comment}	% ロングコメントに必要
\usepackage{arydshln}
\usepackage{emathT}
\usepackage[dvipdfm]{pict2e}
\usepackage{graphicx}
%
% 使用上の注意:
% ・通常の環境に合わせてarydshln、emathTパッケージをインストールすること
% ・セブン-イレブンでのA3プリントに合わせて調節したもの
%
% ・誰が使っても合うようには作っていないので、使用者が使いやすいように枠の大きさや数、配置などを自由に調節してください。
% ・\rule[***zw]{0zw}{***zw} は表の上下の幅を調節してるので、変更するとレイアウトが大きく変わる恐れがあります。
%
\begin{document}
\setlength{\oddsidemargin}{0mm}
\setlength{\evensidemargin}{-10mm}
\setlength{\topmargin}{-5mm}
\setlength{\parindent}{0pt}		%インデントしない
%
% 応募者の基本情報
%     提出日/氏名(苗字・名前・ふりがな)/生年月日(満年齢)/性別
%
\begin{minipage}[!h]{110mm}
	\rule[-3.0zw]{0zw}{4zw}
	{\Large \bf 履歴書 }{ \bf yyyy年 mm月 dd日現在 }\\ 
	\begin{tabularx}{100mm}{Ic|llccI}
		\hlineb
		\multicolumn{1}{Ic|}{\rule[-0.5zw]{0zw}{2zw} \bf ふりがな} & \multicolumn{1}{l}{みょうじ} & \multicolumn{3}{XI}{なまえ}
		\\	\hdashline
		\multicolumn{1}{Ic|}{\rule[-1.5zw]{0zw}{4zw} \bf 氏名} & \multicolumn{1}{l}{{\LARGE 苗字}} & \multicolumn{3}{XI}{{\LARGE 名前}}
		\\	\hline
		\multicolumn{1}{Ic|}{\rule[-0.5zw]{0zw}{2zw} \bf 生年月日} & \multicolumn{2}{X|}{yyyy年mm月dd日生(満N歳)} & \multicolumn{1}{c|}{ \bf 性別} & \multicolumn{1}{cI}{男・女}
		\\	\hlineb
	\end{tabularx}
\end{minipage}
\hfill
% 顔写真
\begin{minipage}[!t]{0.33\textwidth}
	\setlength{\unitlength}{1cm}
	%\begin{picture}(2.5,1) % 写真を別途貼付する場合の枠を描画する
		% 711のA3プリントでは(3.0, 4.0)だと29mm×39mmくらいになる
	%	\put( 0, -1.5){\framebox(3.103, 4.103){ \bf 写真}}	% 3.0cm*(30/29), 4.0cm*(40/39)
	%\end{picture}

	\includegraphics[width=3.0cm]{lenna.eps} % 画像ファイルを使う場合
\end{minipage}
\\
% 住所
% 711のA3プリントでは162mmになる→168mm*(168/162)
\begin{tabularx}{174.222mm}{Ic|lllXI}
	\hlineb
	\rule[-0.5zw]{0zw}{2zw} {\bf ふりがな} & とどうふけん & しくぐん & ちょうそん あざ & \\
	\hdashline
	{\bf 現住所} & 〒XXX - YYYY & & & \\
		   & {\Large 都道府県} & {\Large 市区群} & {\Large 町村 字 xxx番地yyy丁目zzz号} & \\
	\hline
	\multicolumn{1}{Ic|}{ \rule[-0.5zw]{0zw}{2zw} \bf 電話番号} & \multicolumn{4}{XI}{aaa - bbb - cccc} \\
	\hline
	\multicolumn{1}{Ic|}{ \rule[-0.5zw]{0zw}{2zw} \bf 携帯電話} & \multicolumn{4}{XI}{ddd - eeee - ffff} \\
	\hline	
	\multicolumn{1}{Ic|}{ \rule[-0.5zw]{0zw}{2zw} \bf E-mail} & \multicolumn{4}{XI}{ PC: pc-adress } \\
	\multicolumn{1}{Ic|}{ \rule[-0.5zw]{0zw}{2zw} } & \multicolumn{4}{XI}{ 携帯電話: phone-adress } \\
	\hline
	\multicolumn{1}{Ic|}{ \rule[-0.5zw]{0zw}{2zw} \bf  } & \multicolumn{4}{XI}{  } \\ % 空欄。適宜載せたい情報を記入してください(GitHubアカウント等)
	\hlineb
	%他に記す住所がある場合
	\rule[-0.5zw]{0zw}{2zw} {\bf ふりがな} & とどうふけん & しくぐん & ちょうそん あざ & \\
	\hdashline
	{\bf 現住所} & 〒XXX - YYYY & & & \\
		   & {\Large 都道府県} & {\Large 市区群} & {\Large 町村 字 xxx番地yyy丁目zzz号} & \\
	\hline
	\multicolumn{1}{Ic|}{ \rule[-0.5zw]{0zw}{2zw} \bf 電話番号 } & \multicolumn{4}{XI}{ phone-number } \\
	\hlineb
\end{tabularx}
\\
% 学歴・職歴
\begin{tabularx}{174.222mm}{Ic:c|XI}
	\hlineb
	\rule[-0.5zw]{0zw}{2zw}{\bf 年} & {\bf 月} & \multicolumn{1}{cI}{\bf 学歴・職歴} \\	\hline
	\rule[-0.5zw]{0zw}{2zw}  &  & \multicolumn{1}{cI}{\bf 学歴} \\	\hline
	\rule[-0.5zw]{0zw}{2zw} yyyy & mm & highschool 高校 卒業 \\	\hline
	\rule[-0.5zw]{0zw}{2zw} yyyy & mm & university 大学 入学 \\	\hline
	\rule[-0.5zw]{0zw}{2zw} yyyy & mm & university 大学 卒業 \\	\hline
	\rule[-0.5zw]{0zw}{2zw}  &  &  \\	\hline
	\rule[-0.5zw]{0zw}{2zw}  &  & \multicolumn{1}{cI}{\bf 職歴} \\	\hline
	\rule[-0.5zw]{0zw}{2zw} yyyy & mm & company 入社 \\	\hline
	\rule[-0.5zw]{0zw}{2zw} & & \\	\hline
	\rule[-0.5zw]{0zw}{2zw} & &  \\	\hline
	\rule[-0.5zw]{0zw}{2zw} & &  \\	\hline
	\rule[-0.5zw]{0zw}{2zw} & & \multicolumn{1}{rI}{以上} \\	\hline
	\rule[-0.5zw]{0zw}{2zw} & & \\
	\hlineb
\end{tabularx}
\\
% 資格・免許
\begin{tabularx}{174.222mm}{Ic:c|XI}
	\hlineb
	\rule[-0.5zw]{0zw}{2zw}{ \bf 年 }&{ \bf 月 }& \multicolumn{1}{cI}{\bf 資格・免許} \\	\hline
	\rule[-0.5zw]{0zw}{2zw} yyyy & mm & license \\	\hline
	\rule[-0.5zw]{0zw}{2zw} & & \\
	\hlineb
\end{tabularx}
\clearpage
% ===============================ここから2頁目(右側)=============================================================================================
%
\begin{tabularx}{174.222mm}{IXI}
\hlineb
	\multicolumn{1}{IcI}{\rule[-0.5zw]{0zw}{2zw} \bf 自由記述1}
	\\	\hline
	%------------------ここに書く-----------------
	% 空欄のときの見栄えのために改行を入れているので、記入時に適宜削除のこと。
	% コンパイルしたときに、記述量に依らず表内の行数一定にする方法を募集中…
	1 \\
	2 \\
	3 \\
	4 \\
	5 \\
	6 \\
	7 \\
	8 \\
	9 \\
	\hlineb
%
	\multicolumn{1}{IcI}{\rule[-0.5zw]{0zw}{2zw} \bf 自由記述2}
	\\ \hline
	1 \\
	2 \\
	3 \\
	4 \\
	5 \\
	6 \\
	7 \\
	8 \\
	\hlineb
%
	\multicolumn{1}{IcI}{\rule[-0.5zw]{0zw}{2zw} \bf 自由記述3}
	\\ \hline
	1 \\
	2 \\
	3 \\
	4 \\
	5 \\
	6 \\
	7 \\
	\hlineb
%
	\multicolumn{1}{IcI}{\rule[-0.5zw]{0zw}{2zw} \bf 自由記述4}
	\\ \hline
	1 \\
	2 \\
	3 \\
	4 \\
	5 \\
	% ここまで
	\hlineb
%
	\multicolumn{1}{IcI}{\rule[-0.5zw]{0zw}{2zw} \bf 自由記述5}
	\\ \hline
	1 \\
	2 \\
	3 \\
	4 \\
	5 \\
	6 \\
	\hlineb
\end{tabularx}

\end{document}